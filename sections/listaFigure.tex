\documentclass[../rapporto-usabilita.tex]{subfiles}

\begin{document}
\section{Lista delle figure}
	
	In questa sezione viene indicata la corrispondenza tra le figure che appaiono nel presente documento ed il rispettivi file, presenti nella directory \path{\immagini}.
	
	Data la struttura a frame del sito, accedendo tramite la pagina \textit{Home} lo url rimane fisso (in modo da mantenere i frame contenenti il nome del sito, il logo e il menù di navigazione) e cambia solo lo url del frame \textit{Main}. Quindi, per indicare a quale sezione fa riferimento ogni screenshot, è stata indicata la voce del menù di navigazione da cliccare per raggiungerla.
	
	L'unica eccezione riguarda l'esempio di pagina interna acceduta direttamente da SERP, in tal caso è stato indicato lo url.
	
	\begin{center}	

		\begin{tabular}{|c|c|c|}
		\hline
		\textbf{Figura} & \textbf{Nome File}     & \textbf{Sezione}                                 \\ \hline
		1               & home\_focus.jpg        & Home                                             \\ \hline
		2               & nursery.PNG            & \url{http://www.thepuppyranch.com/nursery.htm} \\ \hline
		3               & horizontal\_scroll.jpg & Nursery                                          \\ \hline
		4               & home.png               & Home                                             \\ \hline
		5               & part\_homepage.jpg     & Home                                             \\ \hline
		6               & part\_nursery1.jpg     & Nursery                                          \\ \hline	
		7               & part\_nursery1.jpg     & Nursery                                          \\ \hline
		8               & nursery3.PNG           & Nursery                                          \\ \hline
		9               &  adults.PNG                      &  Heartland Adults                              \\ \hline
		10              &  part\_adults1.jpg                      &     Heartland Adults                                             \\ \hline
		\end{tabular}
	\end{center}
	
\end{document}